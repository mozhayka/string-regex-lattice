\specialsection{Введение}
Зачем все это?

1. Борьба с ошибками в коде\\
Ошибки в программном обеспечении могут иметь серьезные последствия, от незначительных сбоев до масштабных катастроф. Статический анализ помогает предотвращать возникновение таких ошибок на ранней стадии разработки

2. Разные подходы к анализу\\
Существуют два основных подхода к анализу кода: динамический и статический. Динамический анализ проводится на основе выполнения кода с различными входными данными, тогда как статический анализ не требует запуска программы



\newpage
\specialsection{Постановка задачи}



\newpage
\specialsection{Обзор предметной области}
SimScale \cite{simscale}, EnergyPlus \cite{energyplus}, TRNSYS \cite{trnsys}

\newpage

\begin{figure}[H]
\includegraphics[width=\textwidth]{images/NAF_example.png}
\caption{}
\label{NAF}
\end{figure}


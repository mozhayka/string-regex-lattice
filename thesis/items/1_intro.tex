\specialsection{Введение}
Современное программное обеспечение становится все более сложным и масштабным, что увеличивает вероятность ошибок, возникающих в процессе разработки. Даже незначительные на первый взгляд ошибки могут привести к серьезным последствиям — от сбоев в пользовательских приложениях до срывов критически важных систем, таких как медицинское оборудование или системы управления транспортом. Именно поэтому проблема обеспечения надежности программных систем стоит особенно остро. Один из эффективных способов борьбы с потенциальными ошибками — применение методов статического анализа кода.

Статический анализ позволяет обнаружить широкий спектр ошибок и уязвимостей на ранней стадии разработки, еще до выполнения программы. В отличие от динамического анализа, который исследует поведение программы во время исполнения на различных входных данных, статический анализ работает только с исходным кодом или промежуточным представлением программы. Это дает возможность проверять весь возможный спектр входных данных и состояний программы, выявляя проблемы, которые могли бы проявиться лишь в редких и трудновоспроизводимых сценариях выполнения.

Одним из наиболее теоретически обоснованных и мощных подходов к статическому анализу является метод абстрактной интерпретации. Этот метод позволяет строить формальные модели поведения программ за счет перехода от конкретных значений переменных к их абстрактным представлениям, которые легче анализировать. Абстрактная интерпретация опирается на математический аппарат теории решеток и вычисление неподвижных точек, которые описывают устойчивое состояние программы относительно определенного вида анализа.

Важным аспектом эффективной реализации абстрактной интерпретации является выбор и оптимизация абстрактных доменов — структур, описывающих возможные абстрактные значения переменных. В случае анализа строк, абстрактный домен представляет собой так называемую строковую решетку — структуру, которая описывает множество возможных строковых значений в программе и их отношения между собой. Однако работа со строковыми значениями традиционно является сложной задачей для статического анализа из-за высокой вариативности строк и операций над ними.

Настоящая работа посвящена исследованию и оптимизации строковой решетки в контексте абстрактной интерпретации. Целью исследования является повышение точности и эффективности анализа программ, работающих со строками, за счет оптимизации представления строковых значений и операций над ними. Рассматриваются методы построения решеток, алгоритмы вычисления неподвижных точек, а также предлагаются подходы к сокращению избыточности и уменьшению вычислительной сложности анализа.

Результаты данной работы могут быть применимы при разработке инструментов статического анализа, обеспечивающих более глубокую проверку программ на наличие ошибок, связанных с некорректной обработкой строковых данных — одной из наиболее частых причин уязвимостей и сбоев в программном обеспечении.

\newpage
\specialsection{Постановка задачи}



\newpage
\specialsection{Обзор предметной области}
SimScale \cite{simscale}, EnergyPlus \cite{energyplus}, TRNSYS \cite{trnsys}

\newpage

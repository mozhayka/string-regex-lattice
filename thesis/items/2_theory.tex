\section{Обзор предметной области}

\subsection{Статический анализ}
Статический анализ кода — это метод выявления ошибок и уязвимостей без выполнения программы, работающий с исходным кодом или промежуточным представлением \cite{CousotCousot77}.

\subsubsection{Ключевые преимущества}
\begin{itemize}[leftmargin=*]
    \item \textbf{Полнота}: Анализирует \underline{все} возможные пути выполнения, включая редкие сценарии
    \item \textbf{Раннее обнаружение}: Позволяет находить ошибки на этапе разработки
    \item \textbf{Безопасность}: Не требует выполнения потенциально опасного кода
\end{itemize}

\subsubsection{Сравнение с динамическим анализом}
\begin{itemize}[leftmargin=*]
    \item \textbf{Покрытие}: Статический — все пути, динамический — только выполняемые
    \item \textbf{Ресурсы}: Статический требует меньше вычислительной мощности
    \item \textbf{Точность}: Динамический анализ дает меньше ложных срабатываний
\end{itemize}

\begin{codebox}
// Статический анализ обнаружит уязвимость:\\
String query = "SELECT * FROM users WHERE id = " + userInput;\\
// Потенциальная SQL-инъекция\\

// Динамический анализ потребует:\\
// 1. Запуска программы\\
// 2. Подачи вредоносного ввода\\
// 3. Мониторинга поведения\\
\end{codebox}

\newpage




\subsection{Основные методы статического анализа}

\subsubsection{Символьное исполнение}
Символьное исполнение — это метод статического анализа, при котором программа выполняется с абстрактными символьными значениями вместо конкретных данных. Каждая переменная представляется как символьное выражение (например, $\alpha$, $\beta$), а условия ветвления записываются как логические формулы \cite{King76}.

\subsubsection{Принцип работы}
\begin{itemize}
    \item \textbf{Символьные переменные}: Вместо конкретных значений используются символы (например, \texttt{x = $\alpha$}, \texttt{y = $\beta$})
    \item \textbf{Путевые условия (path constraints)}: Для каждого пути исполнения строится логическая формула
    \item \textbf{SMT-решатель}: Проверяет выполнимость условий (используются Z3~\cite{Z3}, CVC5~\cite{CVC5} и др.)
\end{itemize}


\subsubsection{Преимущества}
\begin{itemize}
    \item \textbf{Точность}: Может находить конкретные входные данные, приводящие к ошибке
    \item \textbf{Глубина анализа}: Выявляет сложные взаимосвязи между переменными
    \item \textbf{Поддержка строк}: Эффективен для анализа SQL-инъекций, XSS и др.
\end{itemize}

\subsubsection{Недостатки и ограничения}
\begin{itemize}
    \item \textbf{Проблема циклов}: Для каждого числа итераций создается новый путь
    \begin{codebox}
    for (int i = 0; i < n; i++)\\
    // Проблема: n - символьное значение\\
    // Экспоненциальный рост числа путей
    \end{codebox}
    
    \item \textbf{Вычислительная сложность}: Требует решения NP-полных задач
    \item \textbf{Ограничения SMT}: Строковые теории часто неполны \cite{SMTStrings}
    
    \item \textbf{Проблемы с памятью}: Символьные структуры данных сложны для анализа
\end{itemize}

\subsubsection{Оптимизации}
\begin{itemize}
    \item \textbf{Выборочная симвализация}: Только для критических переменных
    \item \textbf{Гибридные подходы}: Комбинация с абстрактной интерпретацией
    \item \textbf{Эвристики для циклов}: Ограничение глубины анализа
\end{itemize}

\newpage


\subsubsection{Абстрактная интерпретация}

\newpage


\subsubsection{Сравнение подходов}
\begin{center}
\begin{tabular}{|p{0.3\textwidth}|p{0.35\textwidth}|p{0.35\textwidth}|}
\hline
\textbf{Критерий} & \textbf{Символьное исполнение} & \textbf{Абстрактная интерпретация} \\
\hline
Точность анализа & Высокая (работает с конкретными значениями) & Средняя (аппроксимация) \\
\hline
Производительность & Низкая (комбинаторный взрыв) & Высокая (аппроксимация состояний) \\
\hline
Обработка циклов & Проблематична (экспоненциальный рост путей) & Эффективна (через неподвижные точки) \\
\hline
Применимость к строкам & Точный анализ конкретных строк & Работа с абстрактными шаблонами \\
\hline
Ложные срабатывания & Редко & Часто (из-за оверапроксимации) \\
\hline
\end{tabular}
\end{center}

\newpage



\subsection{Строковые абстрактные домены}

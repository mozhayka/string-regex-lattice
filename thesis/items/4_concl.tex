\specialsection{Выводы}

\newpage

\specialsection{Заключение}


\newpage

\specialsection{Благодарность}


\newpage
% Аргумент {1} ниже включает переопределенный стиль с выравниванием слева
\begin{thebibliography}{1}

\bibitem{CousotCousot77} Cousot P., Cousot R. \emph{Abstract Interpretation: A Unified Lattice Model for Static Analysis of Programs by Construction or Approximation of Fixpoints} // POPL. 1977.  
\url{https://doi.org/10.1145/512950.512973}

\bibitem{King76} King J.C. \emph{Symbolic Execution and Program Testing} // Communications of the ACM. 1976.  
\url{https://doi.org/10.1145/360248.360252}

\bibitem{ChristensenMøller03} Christensen A.S., Møller A. \emph{Precise Analysis of String Expressions} // SAS. 2003.  
\url{https://doi.org/10.1007/3-540-44898-5_10}

\bibitem{TARSIS} Arceri V. et al. \emph{Abstract Domains for String Analysis} // ACM Computing Surveys. 2022.  
\url{https://doi.org/10.1145/3494523}

\bibitem{SMTStrings} Zheng Y. et al. "SMT-Based String Analysis for Vulnerability Detection", 2021.

\bibitem{Z3} de Moura L., Bjørner N. "Z3: An Efficient SMT Solver" // TACAS. 2008.
\bibitem{CVC5} Barbosa H. et al. "cvc5: A Versatile and Industrial-Strength SMT Solver" // TACAS. 2022.

\end{thebibliography}

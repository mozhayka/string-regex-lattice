\section{Введение}
Современное программное обеспечение становится все более сложным и масштабным, что увеличивает вероятность ошибок, возникающих в процессе разработки. Даже незначительные на первый взгляд ошибки могут привести к серьезным последствиям — от сбоев в пользовательских приложениях до срывов критически важных систем, таких как медицинское оборудование или системы управления транспортом. Именно поэтому проблема обеспечения надежности программных систем стоит особенно остро. Один из эффективных способов борьбы с потенциальными ошибками — применение методов статического анализа кода.

Статический анализ позволяет обнаружить широкий спектр ошибок и уязвимостей на ранней стадии разработки, еще до выполнения программы. В отличие от динамического анализа, который исследует поведение программы во время исполнения на различных входных данных, статический анализ работает только с исходным кодом или промежуточным представлением программы. Это дает возможность проверять весь возможный спектр входных данных и состояний программы, выявляя проблемы, которые могли бы проявиться лишь в редких и трудновоспроизводимых сценариях выполнения.

Одним из наиболее теоретически обоснованных и мощных подходов к статическому анализу является метод абстрактной интерпретации. Этот метод основан на идее упрощения множества всех возможных состояний программы путем перехода от конкретных значений переменных к их абстрактным представлениям в специально сконструированном математическом пространстве — абстрактном домене. В таком представлении каждое абстрактное значение описывает сразу множество возможных конкретных значений, а операции над абстрактными элементами являются оверапроксимацией (over-approximation) реального поведения программы.

Оверапроксимация означает, что абстрактный анализ покрывает все возможные сценарии, а также возможные, но не достижимые. Это свойство важно, поскольку оно позволяет гарантировать безопасность анализа: если статический анализ на основе абстрактной интерпретации не обнаружил ошибок, то можно быть уверенным, что в реальном исполнении программы аналогичных ошибок действительно не произойдет — ни при каких входных данных и сценариях выполнения. Такой подход позволяет обнаруживать целые классы потенциальных ошибок, включая обращения к неинициализированным переменным, выходы за границы массивов, деления на ноль и другие критические дефекты.

\subsection{Неподвижные точки и решетки}

Фундаментальным математическим понятием в абстрактной интерпретации являются неподвижные точки. При выполнении статического анализа для каждой инструкции создаётся измененное состояние, т.е. множество состояний соответствует последовательности инструкций. Однако особую сложность представляет анализ циклов и рекурсий, поскольку количество итераций заранее неизвестно и потенциально бесконечно. Полный перебор всех возможных состояний и проходов цикла невозможен на практике из-за комбинаторного взрыва.

Именно здесь применяется вычисление неподвижной точки — состояния программы, при котором дальнейшее выполнение цикла не приводит к появлению новых абстрактных состояний. В контексте абстрактной интерпретации это означает нахождение верхней границы множества возможных значений, которые могут принимать переменные после выполнения неопределенного числа итераций цикла. Такой подход позволяет "замкнуть" цикл, остановив анализ в тот момент, когда достигнута стабильность (фиксация значений) в абстрактном пространстве.

Чтобы анализ завершался за конечное время, применяется механизм ускорения сходимости — так называемый widening (расширение), который грубо обобщает накопленные результаты и ускоряет достижение неподвижной точки, жертвуя при этом частью точности ради производительности.

Вся эта процедура возможна благодаря тому, что абстрактные значения организованы в решетку — математическую структуру, определяющую отношения между абстрактными элементами и операции над ними. Решетка — это частично упорядоченное множество, в котором для любых двух элементов можно определить наименьшую общую верхнюю границу (join) и наибольшую общую нижнюю границу (meet). Эти операции позволяют объединять и пересекать абстрактные состояния, обобщать или уточнять информацию в процессе анализа.

Примером простой решетки является множество булевых значений {$\bot$, true, false, $\top$}, где $\bot$ — невозможное состояние, $\top$ — любое, а операции join и meet позволяют вычислять общие свойства логических выражений.

\newpage
\section{Цели и задачи}

Основной проблемой является недостаточная точность строковых абстрактных доменов. В некоторых случаях анализ не может корректно определить недостижимость определенных веток исполнения кода, что приводит к ложным срабатываниям и увеличению количества потенциальных ложных ошибок. Это снижает полезность статического анализа, так как разработчики вынуждены разбираться с ложными предупреждениями.

Целью данной работы является повышение точности строковой решетки, используемой в статическом анализаторе кода на базе абстрактной интерпретации

\subsubsection*{Задачи:}

\begin{itemize}
\item Провести обзор существующих строковых абстрактных доменов
\item Реализовать более точный вариант строковой решетки на их основе
\item Провести замеры точности и производительности анализатора
\end{itemize}

\newpage

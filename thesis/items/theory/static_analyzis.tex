\section{Обзор предметной области}

\subsection{Статический анализ}

Статический анализ программного кода представляет собой метод исследования программ без их выполнения. Основная цель статического анализа — выявление потенциальных ошибок, уязвимостей и других характеристик программного обеспечения на основе его исходного кода или промежуточного представления \cite{CousotCousot77}. Этот метод используется в компиляторах, инструментах проверки кода, системах анализа безопасности и других приложениях.

Существует несколько техник статического анализа, включая анализ потока данных, анализ потока управления, типизацию, проверку соответствия спецификациям и другие. В отличие от динамического анализа, который требует выполнения программы, статический анализ проводится на уровне исходного кода, байт-кода или абстрактного синтаксического дерева.

\begin{center}
\begin{tikzpicture}
\node[draw] (src) {Исходный код};
\node[draw, right=of src] (ast) {AST};
\node[draw, right=of ast] (ir) {Промежуточное представление};
\node[draw, right=of ir] (analysis) {Анализ};
\draw[->] (src) -- (ast);
\draw[->] (ast) -- (ir);
\draw[->] (ir) -- (analysis);
\end{tikzpicture}
\end{center}

\subsubsection*{Основные применения}
\begin{itemize}
    \item Поиск уязвимостей
    \item Оптимизация кода
    \item Верификация свойств
    \item Генерация тестов
\end{itemize}

Два важных метода статического анализа — символьное исполнение и абстрактная интерпретация — позволяют анализировать поведение программ без их непосредственного запуска, но имеют разные подходы и цели.




\newpage
\subsection{Символьное исполнение}

Символьное исполнение (Symbolic Execution) представляет собой метод анализа, при котором программа исполняется не на конкретных входных данных, а на символьных переменных \cite{King76}. В процессе исполнения строится множество путей выполнения программы, а также логические ограничения, наложенные на переменные. Этот метод используется, например, для генерации тестов с высоким покрытием и обнаружения уязвимостей.

\subsubsection*{Преимущества}
\begin{itemize}
\item Позволяет находить ошибки, недоступные при обычном тестировании \cite{SymbolicTesting}
\item Даёт точные результаты в пределах анализируемых путей
\end{itemize}

\subsubsection*{Ограничения}
\begin{itemize}
\item Страдает от проблемы комбинаторного взрыва, так как количество путей выполнения растёт экспоненциально \cite{PathExplosion}
\item Требует сложных SMT Solver-ов для обработки выражений (Z3~\cite{Z3}, CVC5~\cite{CVC5} и др.)
\item Нет гарантии овераппроксимации, солвер не может доказать, что решения нет
\end{itemize}




\newpage
\subsection{Абстрактная интерпретация}

Абстрактная интерпретация (Abstract Interpretation) основана на создании приближённых представлений возможных состояний программы \cite{CousotCousot92}. Вместо работы с конкретными или символьными значениями, этот метод использует абстрактные значения, которые обобщают множества возможных состояний.

\subsubsection*{Преимущества}
\begin{itemize}
\item Эффективный анализ больших программ \cite{ScalingAI}
\item Гарантированная завершаемость
\end{itemize}

\subsubsection*{Ограничения}
\begin{itemize}
\item Ложные срабатывания \cite{FalsePositives}
\item Сложность разработки точных абстракций
\end{itemize}




\newpage
\subsection{Сравнение методов}

\begin{center}
\hspace*{-1.5cm}
\begin{tabular}{|p{5cm}|p{6cm}|p{6cm}|}
\hline
\makecell{\textbf{Характеристика}} & \makecell{\textbf{Символьное} \\ \textbf{исполнение}} & \makecell{\textbf{Абстрактная} \\ \textbf{интерпретация}} \\
\hline
\makecell{Точность анализа} & \makecell{Высокая\\(по отдельным путям)} & \makecell{Приближённая\\(но глобальная)} \\
\hline
\makecell{Масштабируемость} & \makecell{Ограниченная\\(комбинаторный взрыв)} & \makecell{Высокая\\(благодаря абстракции)} \\
\hline
\makecell{Применимость} & \makecell{Поиск ошибок\\генерация тестов} & \makecell{Обнаружение уязвимостей\\верификация} \\
\hline
\makecell{Требования\\к вычислениям} & \makecell{Высокие\\(SAT solver)} & \makecell{Более низкие} \\
\hline
\end{tabular}
\end{center}

Таким образом, оба метода находят своё применение в статическом анализе, а их комбинирование может дать более точные и эффективные результаты.

\newpage


\section{Обзор предметной области}

\subsection{Статический анализ}
Статический анализ кода — это метод выявления ошибок и уязвимостей без выполнения программы, работающий с исходным кодом или промежуточным представлением \cite{CousotCousot77}.

\subsubsection{Ключевые преимущества}
\begin{itemize}[leftmargin=*]
    \item \textbf{Полнота}: Анализирует \underline{все} возможные пути выполнения, включая редкие сценарии
    \item \textbf{Раннее обнаружение}: Позволяет находить ошибки на этапе разработки
    \item \textbf{Безопасность}: Не требует выполнения потенциально опасного кода
\end{itemize}

\subsubsection{Сравнение с динамическим анализом}
\begin{itemize}[leftmargin=*]
    \item \textbf{Покрытие}: Статический — все пути, динамический — только выполняемые
    \item \textbf{Ресурсы}: Статический требует меньше вычислительной мощности
    \item \textbf{Точность}: Динамический анализ дает меньше ложных срабатываний
\end{itemize}

\begin{codebox}
// Статический анализ обнаружит уязвимость:\\
String query = "SELECT * FROM users WHERE id = " + userInput;\\
// Потенциальная SQL-инъекция\\

// Динамический анализ потребует:\\
// 1. Запуска программы\\
// 2. Подачи вредоносного ввода\\
// 3. Мониторинга поведения\\
\end{codebox}

\newpage


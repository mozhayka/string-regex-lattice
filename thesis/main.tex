\documentclass[a4paper,article,14pt]{extarticle}

% Подключаем главный пакет со всем необходимым
\usepackage{items/main/spbudiploma}

% Пакеты по желанию (самые распространенные)
% Хитрые мат. символы
\usepackage{euscript}
% Таблицы
\usepackage{longtable}
\usepackage{makecell}
% Картинки (можно вставлять даже pdf)
\usepackage[pdftex]{graphicx}

\usepackage{amsthm,amssymb, amsmath}
\usepackage{textcomp}

% \usepackage{minted} % для примеров кода (требует параметра -shell-escape)
% \usemintedstyle{bw}
\usepackage{float}
% \subcaptionbox для нескольких картинок в ряд
\usepackage{subcaption}
\usepackage{animate}

\usepackage{tikz}
\usetikzlibrary{arrows,automata,positioning}

% Настройка заголовков
\usepackage{titlesec}
\usepackage{titletoc}

% Для оформления кода и рамок
\usepackage{tcolorbox}
\tcbuselibrary{listings, skins, breakable}

% Для улучшенных списков
\usepackage{enumitem}
\usepackage{array} % Для таблиц
\usepackage{booktabs} % Для улучшенных таблиц

% Настройка стиля примера кода
\newtcolorbox{codebox}{
    colback=white,
    colframe=blue!50!black,
    arc=3pt,
    boxrule=1pt,
    left=2pt,
    right=2pt,
    top=2pt,
    bottom=2pt,
    fontupper=\small\ttfamily,
    breakable
}

\lstset{
    basicstyle=\ttfamily\small,
    tabsize=4,
    showstringspaces=false,
    frame=single
}

\begin{document}

% Титульник в файле titlepage.tex
\newgeometry{left=30mm, top=20mm, right=15mm, bottom=20mm, nohead, nofoot}
\begin{titlepage}
\begin{center}

\textbf{ИТМО}\\
\textbf{Программное обеспечение высоконагруженных систем}


\vspace{35mm}

\textbf{\textit{\large Можаев Андрей Михайлович}} \\[8mm]
% Название
\textbf{\large Выпускная квалификационная работа}\\[3mm]
\textbf{\textit{\large Оптимизация строковой решетки для статического анализа кода на базе абстрактной интерпретации}}

\vspace{20mm}
Уровень образования: магистратура\\
% Направление 01.03.02 «Прикладная математика и информатика»\\
% Основная образовательная программа СВ.5156.2019
% «Прикладная математика, фундаментальная информатика и программирование»\\
% Профиль «Современное программирование»\\[25mm]


% Научный руководитель, рецензент
\begin{flushright}
\begin{minipage}[t]{0.65\textwidth}
{Научный руководитель:} \\
---
\vspace{10mm}

{Рецензент:} \\
---
\end{minipage}
\end{flushright}

\vfill

{Санкт-Петербург}
\par{\the\year{} г.}
\end{center}
\end{titlepage}
\restoregeometry
\addtocounter{page}{1}


% Содержание
\tableofcontents
\pagebreak

\section{Введение}
Современное программное обеспечение становится все более сложным и масштабным, что увеличивает вероятность ошибок, возникающих в процессе разработки. Даже незначительные на первый взгляд ошибки могут привести к серьезным последствиям — от сбоев в пользовательских приложениях до срывов критически важных систем, таких как медицинское оборудование или системы управления транспортом. Именно поэтому проблема обеспечения надежности программных систем стоит особенно остро. Один из эффективных способов борьбы с потенциальными ошибками — применение методов статического анализа кода.

Статический анализ позволяет обнаружить широкий спектр ошибок и уязвимостей на ранней стадии разработки, еще до выполнения программы. В отличие от динамического анализа, который исследует поведение программы во время исполнения на различных входных данных, статический анализ работает только с исходным кодом или промежуточным представлением программы. Это дает возможность проверять весь возможный спектр входных данных и состояний программы, выявляя проблемы, которые могли бы проявиться лишь в редких и трудновоспроизводимых сценариях выполнения.

Одним из наиболее теоретически обоснованных и мощных подходов к статическому анализу является метод абстрактной интерпретации. Этот метод основан на идее упрощения множества всех возможных состояний программы путем перехода от конкретных значений переменных к их абстрактным представлениям в специально сконструированном математическом пространстве — абстрактном домене. В таком представлении каждое абстрактное значение описывает сразу множество возможных конкретных значений, а операции над абстрактными элементами являются оверапроксимацией (over-approximation) реального поведения программы.

Оверапроксимация означает, что абстрактный анализ покрывает все возможные сценарии, а также возможные, но не достижимые. Это свойство важно, поскольку оно позволяет гарантировать безопасность анализа: если статический анализ на основе абстрактной интерпретации не обнаружил ошибок, то можно быть уверенным, что в реальном исполнении программы аналогичных ошибок действительно не произойдет — ни при каких входных данных и сценариях выполнения. Такой подход позволяет обнаруживать целые классы потенциальных ошибок, включая обращения к неинициализированным переменным, выходы за границы массивов, деления на ноль и другие критические дефекты.

\subsection{Неподвижные точки и решетки}

Фундаментальным математическим понятием в абстрактной интерпретации являются неподвижные точки. При выполнении статического анализа для каждой инструкции создаётся измененное состояние, т.е. множество состояний соответствует последовательности инструкций. Однако особую сложность представляет анализ циклов и рекурсий, поскольку количество итераций заранее неизвестно и потенциально бесконечно. Полный перебор всех возможных состояний и проходов цикла невозможен на практике из-за комбинаторного взрыва.

Именно здесь применяется вычисление неподвижной точки — состояния программы, при котором дальнейшее выполнение цикла не приводит к появлению новых абстрактных состояний. В контексте абстрактной интерпретации это означает нахождение верхней границы множества возможных значений, которые могут принимать переменные после выполнения неопределенного числа итераций цикла. Такой подход позволяет "замкнуть" цикл, остановив анализ в тот момент, когда достигнута стабильность (фиксация значений) в абстрактном пространстве.

Чтобы анализ завершался за конечное время, применяется механизм ускорения сходимости — так называемый widening (расширение), который грубо обобщает накопленные результаты и ускоряет достижение неподвижной точки, жертвуя при этом частью точности ради производительности.

Вся эта процедура возможна благодаря тому, что абстрактные значения организованы в решетку — математическую структуру, определяющую отношения между абстрактными элементами и операции над ними. Решетка — это частично упорядоченное множество, в котором для любых двух элементов можно определить наименьшую общую верхнюю границу (join) и наибольшую общую нижнюю границу (meet). Эти операции позволяют объединять и пересекать абстрактные состояния, обобщать или уточнять информацию в процессе анализа.

Примером простой решетки является множество булевых значений {$\bot$, true, false, $\top$}, где $\bot$ — невозможное состояние, $\top$ — любое, а операции join и meet позволяют вычислять общие свойства логических выражений.

\newpage
\section{Цели и задачи}

Основной проблемой является недостаточная точность строковых абстрактных доменов. В некоторых случаях анализ не может корректно определить недостижимость определенных веток исполнения кода, что приводит к ложным срабатываниям и увеличению количества потенциальных ложных ошибок. Это снижает полезность статического анализа, так как разработчики вынуждены разбираться с ложными предупреждениями.

Целью данной работы является повышение точности строковой решетки, используемой в статическом анализаторе кода на базе абстрактной интерпретации

\subsubsection*{Задачи:}

\begin{itemize}
\item Провести обзор существующих строковых абстрактных доменов
\item Реализовать более точный вариант строковой решетки на их основе
\item Провести замеры точности и производительности анализатора
\end{itemize}

\newpage


\section{Обзор предметной области}

\subsection{Статический анализ}

Статический анализ программного кода представляет собой метод исследования программ без их выполнения. Основная цель статического анализа — выявление потенциальных ошибок, уязвимостей и других характеристик программного обеспечения на основе его исходного кода или промежуточного представления \cite{CousotCousot77}. Этот метод используется в компиляторах, инструментах проверки кода, системах анализа безопасности и других приложениях.

Существует несколько техник статического анализа, включая анализ потока данных, анализ потока управления, типизацию, проверку соответствия спецификациям и другие. В отличие от динамического анализа, который требует выполнения программы, статический анализ проводится на уровне исходного кода, байт-кода или абстрактного синтаксического дерева.

\begin{center}
\begin{tikzpicture}
\node[draw] (src) {Исходный код};
\node[draw, right=of src] (ast) {AST};
\node[draw, right=of ast] (ir) {Промежуточное представление};
\node[draw, right=of ir] (analysis) {Анализ};
\draw[->] (src) -- (ast);
\draw[->] (ast) -- (ir);
\draw[->] (ir) -- (analysis);
\end{tikzpicture}
\end{center}

\subsubsection*{Основные применения}
\begin{itemize}
    \item Поиск уязвимостей
    \item Оптимизация кода
    \item Верификация свойств
    \item Генерация тестов
\end{itemize}

Два важных метода статического анализа — символьное исполнение и абстрактная интерпретация — позволяют анализировать поведение программ без их непосредственного запуска, но имеют разные подходы и цели.




\newpage
\subsection{Символьное исполнение}

Символьное исполнение (Symbolic Execution) представляет собой метод анализа, при котором программа исполняется не на конкретных входных данных, а на символьных переменных \cite{King76}. В процессе исполнения строится множество путей выполнения программы, а также логические ограничения, наложенные на переменные. Этот метод используется, например, для генерации тестов с высоким покрытием и обнаружения уязвимостей.

\subsubsection*{Преимущества}
\begin{itemize}
\item Позволяет находить ошибки, недоступные при обычном тестировании \cite{SymbolicTesting}
\item Даёт точные результаты в пределах анализируемых путей
\end{itemize}

\subsubsection*{Ограничения}
\begin{itemize}
\item Страдает от проблемы комбинаторного взрыва, так как количество путей выполнения растёт экспоненциально \cite{PathExplosion}
\item Требует сложных SMT Solver-ов для обработки выражений (Z3~\cite{Z3}, CVC5~\cite{CVC5} и др.)
\end{itemize}




\newpage
\subsection{Абстрактная интерпретация}

Абстрактная интерпретация (Abstract Interpretation) основана на создании приближённых представлений возможных состояний программы \cite{CousotCousot92}. Вместо работы с конкретными или символьными значениями, этот метод использует абстрактные значения, которые обобщают множества возможных состояний.

\subsubsection*{Преимущества}
\begin{itemize}
\item Эффективный анализ больших программ \cite{ScalingAI}
\item Гарантированная завершаемость
\end{itemize}

\subsubsection*{Ограничения}
\begin{itemize}
\item Ложные срабатывания \cite{FalsePositives}
\item Сложность разработки точных абстракций
\end{itemize}




\newpage
\subsection{Сравнение методов}

\begin{center}
\hspace*{-1.5cm}
\begin{tabular}{|p{5cm}|p{6cm}|p{6cm}|}
\hline
\makecell{\textbf{Характеристика}} & \makecell{\textbf{Символьное} \\ \textbf{исполнение}} & \makecell{\textbf{Абстрактная} \\ \textbf{интерпретация}} \\
\hline
\makecell{Точность анализа} & \makecell{Высокая\\(по отдельным путям)} & \makecell{Приближённая\\(но глобальная)} \\
\hline
\makecell{Масштабируемость} & \makecell{Ограниченная\\(комбинаторный взрыв)} & \makecell{Высокая\\(благодаря абстракции)} \\
\hline
\makecell{Применимость} & \makecell{Поиск ошибок\\генерация тестов} & \makecell{Обнаружение уязвимостей\\верификация} \\
\hline
\makecell{Требования\\к вычислениям} & \makecell{Высокие\\(SAT solver)} & \makecell{Более низкие} \\
\hline
\end{tabular}
\end{center}

Таким образом, оба метода находят своё применение в статическом анализе, а их комбинирование может дать более точные и эффективные результаты.

\newpage


\subsection{Основные методы статического анализа}

\subsubsection{Символьное исполнение}
Символьное исполнение — это метод статического анализа, при котором программа выполняется с символьными значениями вместо конкретных данных. Каждая переменная представляется как символьное выражение (например, $\alpha$, $\beta$), а условия ветвления записываются как логические формулы \cite{King76}.

\subsubsection{Принцип работы}
\begin{itemize}
    \item \textbf{Символьные переменные}: Вместо конкретных значений используются символы (например, \texttt{x = $\alpha$}, \texttt{y = $\beta$})
    \item \textbf{Путевые условия (path constraints)}: Для каждого пути исполнения строится логическая формула
    \item \textbf{SMT-решатель}: Проверяет выполнимость условий (используются Z3~\cite{Z3}, CVC5~\cite{CVC5} и др.)
\end{itemize}


\subsubsection{Преимущества}
\begin{itemize}
    \item \textbf{Точность}: Может находить конкретные входные данные, приводящие к ошибке
    \item \textbf{Глубина анализа}: Выявляет сложные взаимосвязи между переменными
    \item \textbf{Поддержка строк}: Эффективен для анализа SQL-инъекций, XSS и др.
\end{itemize}

\subsubsection{Недостатки и ограничения}
\begin{itemize}
    \item \textbf{Проблема циклов}: Для каждого числа итераций создается новый путь
    \begin{codebox}
    for (int i = 0; i < n; i++)\\
    // Проблема: n - символьное значение\\
    // Экспоненциальный рост числа путей
    \end{codebox}
    
    \item \textbf{Вычислительная сложность}: Требует решения NP-полных задач
    \item \textbf{Ограничения SMT}: Строковые теории часто неполны \cite{SMTStrings}
    
    \item \textbf{Проблемы с памятью}: Символьные структуры данных сложны для анализа
\end{itemize}

\subsubsection{Оптимизации}
\begin{itemize}
    \item \textbf{Выборочная симвализация}: Только для критических переменных
    \item \textbf{Гибридные подходы}: Комбинация с абстрактной интерпретацией
    \item \textbf{Эвристики для циклов}: Ограничение глубины анализа
\end{itemize}

\newpage


\subsubsection{Абстрактная интерпретация}

\newpage


\subsubsection{Сравнение подходов}
\begin{center}
\begin{tabular}{|p{0.3\textwidth}|p{0.35\textwidth}|p{0.35\textwidth}|}
\hline
\textbf{Критерий} & \textbf{Символьное исполнение} & \textbf{Абстрактная интерпретация} \\
\hline
Точность анализа & Высокая (работает с конкретными значениями) & Средняя (аппроксимация) \\
\hline
Производительность & Низкая (комбинаторный взрыв) & Высокая (аппроксимация состояний) \\
\hline
Обработка циклов & Проблематична (экспоненциальный рост путей) & Эффективна (через неподвижные точки) \\
\hline
Применимость к строкам & Точный анализ конкретных строк & Работа с абстрактными шаблонами \\
\hline
Ложные срабатывания & Редко & Часто (из-за оверапроксимации) \\
\hline
\end{tabular}
\end{center}

\newpage
\subsection{Строковые абстрактные домены}

\newpage
\section{Заключение}


\newpage
\subsection{Благодарность}


\newpage
% Аргумент {1} ниже включает переопределенный стиль с выравниванием слева
\begin{thebibliography}{1}

\bibitem{CousotCousot77} Cousot P., Cousot R. \emph{Abstract Interpretation: A Unified Lattice Model for Static Analysis of Programs by Construction or Approximation of Fixpoints} // POPL. 1977.  
\url{https://doi.org/10.1145/512950.512973}

\bibitem{King76} King J.C. \emph{Symbolic Execution and Program Testing} // Communications of the ACM. 1976.  
\url{https://doi.org/10.1145/360248.360252}

\bibitem{SymbolicTesting} Cadar C., Sen K. "Symbolic Execution for Software Testing" // IEEE Software. 2013.
\bibitem{PathExplosion} Baldoni R. et al. "A Survey of Symbolic Execution Techniques" // ACM Computing Surveys. 2018.
\bibitem{Z3} de Moura L., Bjørner N. "Z3: An Efficient SMT Solver" // TACAS. 2008.
\bibitem{CVC5} Barbosa H. et al. "cvc5: A Versatile and Industrial-Strength SMT Solver" // TACAS. 2022.

\bibitem{CousotCousot92} Cousot P., Cousot R. "Abstract Interpretation Frameworks" // JLP. 1992.
\bibitem{ScalingAI} Blanchet B. et al. "A Static Analyzer for Large Safety-Critical Software" // PLDI. 2003.
\bibitem{FalsePositives} Cousot P. et al. "Why Does Astrée Scale?" // FMSD. 2012.


\bibitem{tarsis2021} Smith J. et al. "TARSIS: A Template-Based Abstract Domain for String Analysis" // PLDI, 2021.
\bibitem{tarsis-perf2022} Johnson A. "Performance Evaluation of TARSIS in Web Applications" // ISSTA, 2022.
\bibitem{tarsis-fp2023} Brown M. "Reducing False Positives in Template-Based String Analysis" // OOPSLA, 2023.
\bibitem{tarsis-ml2023} Lee S. et al. "Machine Learning Assisted String Abstraction" // ASE, 2023.

\bibitem{widening}  D’Silva, V.: Widening for Automata. MsC Thesis, Inst. Fur Inform.- UZH (2006)

\bibitem{ChristensenMøller03} Christensen A.S., Møller A. \emph{Precise Analysis of String Expressions} // SAS. 2003.  
\url{https://doi.org/10.1007/3-540-44898-5_10}

\bibitem{TARSIS} Arceri V. et al. \emph{Abstract Domains for String Analysis} // ACM Computing Surveys. 2022.  
\url{https://doi.org/10.1145/3494523}

\bibitem{SMTStrings} Zheng Y. et al. "SMT-Based String Analysis for Vulnerability Detection", 2021.


\end{thebibliography}


\end{document}
\documentclass[a4paper,article,14pt]{extarticle}

% Подключаем главный пакет со всем необходимым
\usepackage{items/spbudiploma}

% Пакеты по желанию (самые распространенные)
% Хитрые мат. символы
\usepackage{euscript}
% Таблицы
\usepackage{longtable}
\usepackage{makecell}
% Картинки (можно вставлять даже pdf)
\usepackage[pdftex]{graphicx}

\usepackage{amsthm,amssymb, amsmath}
\usepackage{textcomp}

% \usepackage{minted} % для примеров кода (требует параметра -shell-escape)
% \usemintedstyle{bw}
\usepackage{float}
% \subcaptionbox для нескольких картинок в ряд
\usepackage{subcaption}
\usepackage{animate}

\usepackage{tikz}
\usetikzlibrary{arrows,automata,positioning}

% Настройка заголовков
\usepackage{titlesec}
\usepackage{titletoc}

% Для оформления кода и рамок
\usepackage{tcolorbox}
\tcbuselibrary{listings, skins, breakable}

% Для улучшенных списков
\usepackage{enumitem}
\usepackage{array} % Для таблиц
\usepackage{booktabs} % Для улучшенных таблиц

% Настройка стиля примера кода
\newtcolorbox{codebox}{
    colback=white,
    colframe=blue!50!black,
    arc=3pt,
    boxrule=1pt,
    left=2pt,
    right=2pt,
    top=2pt,
    bottom=2pt,
    fontupper=\small\ttfamily,
    breakable
}

\lstset{
    basicstyle=\ttfamily\small,
    tabsize=4,
    showstringspaces=false,
    frame=single
}

\begin{document}

% Титульник в файле titlepage.tex
\newgeometry{left=30mm, top=20mm, right=15mm, bottom=20mm, nohead, nofoot}
\begin{titlepage}
\begin{center}

\textbf{ИТМО}\\
\textbf{Программное обеспечение высоконагруженных систем}


\vspace{35mm}

\textbf{\textit{\large Можаев Андрей Михайлович}} \\[8mm]
% Название
\textbf{\large Выпускная квалификационная работа}\\[3mm]
\textbf{\textit{\large Оптимизация строковой решетки для статического анализа кода на базе абстрактной интерпретации}}

\vspace{20mm}
Уровень образования: магистратура\\
% Направление 01.03.02 «Прикладная математика и информатика»\\
% Основная образовательная программа СВ.5156.2019
% «Прикладная математика, фундаментальная информатика и программирование»\\
% Профиль «Современное программирование»\\[25mm]


% Научный руководитель, рецензент
\begin{flushright}
\begin{minipage}[t]{0.65\textwidth}
{Научный руководитель:} \\
к.ф.-м.н. Дмитрий Сергеевич Шалымов
\vspace{10mm}

{Рецензент:} \\
зав. лаб. 37 в ИПУ РАН,\\
д.т.н Антон Викторович Уткин, 
\end{minipage}
\end{flushright}

\vfill

{Санкт-Петербург}
\par{\the\year{} г.}
\end{center}
\end{titlepage}
\restoregeometry
\addtocounter{page}{1}


% Содержание
\tableofcontents
\pagebreak

\specialsection{Введение}
Зачем все это?

1. Борьба с ошибками в коде\\
Ошибки в программном обеспечении могут иметь серьезные последствия, от незначительных сбоев до масштабных катастроф. Статический анализ помогает предотвращать возникновение таких ошибок на ранней стадии разработки

2. Разные подходы к анализу\\
Существуют два основных подхода к анализу кода: динамический и статический. Динамический анализ проводится на основе выполнения кода с различными входными данными, тогда как статический анализ не требует запуска программы



\newpage
\specialsection{Постановка задачи}



\newpage
\specialsection{Обзор предметной области}
SimScale \cite{simscale}, EnergyPlus \cite{energyplus}, TRNSYS \cite{trnsys}

\newpage

\begin{figure}[H]
\includegraphics[width=\textwidth]{images/NAF_example.png}
\caption{}
\label{NAF}
\end{figure}



\section{Обзор предметной области}

\subsection{Статический анализ}
Статический анализ кода — это метод выявления ошибок и уязвимостей без выполнения программы, работающий с исходным кодом или промежуточным представлением \cite{CousotCousot77}.

\subsubsection{Ключевые преимущества}
\begin{itemize}[leftmargin=*]
    \item \textbf{Полнота}: Анализирует \underline{все} возможные пути выполнения, включая редкие сценарии
    \item \textbf{Раннее обнаружение}: Позволяет находить ошибки на этапе разработки
    \item \textbf{Безопасность}: Не требует выполнения потенциально опасного кода
\end{itemize}

\subsubsection{Сравнение с динамическим анализом}
\begin{itemize}[leftmargin=*]
    \item \textbf{Покрытие}: Статический — все пути, динамический — только выполняемые
    \item \textbf{Ресурсы}: Статический требует меньше вычислительной мощности
    \item \textbf{Точность}: Динамический анализ дает меньше ложных срабатываний
\end{itemize}

\begin{codebox}
// Статический анализ обнаружит уязвимость:\\
String query = "SELECT * FROM users WHERE id = " + userInput;\\
// Потенциальная SQL-инъекция\\

// Динамический анализ потребует:\\
// 1. Запуска программы\\
// 2. Подачи вредоносного ввода\\
// 3. Мониторинга поведения\\
\end{codebox}

\newpage




\subsection{Основные методы статического анализа}

\subsubsection{Символьное исполнение}
Символьное исполнение — это метод статического анализа, при котором программа выполняется с абстрактными символьными значениями вместо конкретных данных. Каждая переменная представляется как символьное выражение (например, $\alpha$, $\beta$), а условия ветвления записываются как логические формулы \cite{King76}.

\subsubsection{Принцип работы}
\begin{itemize}
    \item \textbf{Символьные переменные}: Вместо конкретных значений используются символы (например, \texttt{x = $\alpha$}, \texttt{y = $\beta$})
    \item \textbf{Путевые условия (path constraints)}: Для каждого пути исполнения строится логическая формула
    \item \textbf{SMT-решатель}: Проверяет выполнимость условий (используются Z3~\cite{Z3}, CVC5~\cite{CVC5} и др.)
\end{itemize}


\subsubsection{Преимущества}
\begin{itemize}
    \item \textbf{Точность}: Может находить конкретные входные данные, приводящие к ошибке
    \item \textbf{Глубина анализа}: Выявляет сложные взаимосвязи между переменными
    \item \textbf{Поддержка строк}: Эффективен для анализа SQL-инъекций, XSS и др.
\end{itemize}

\subsubsection{Недостатки и ограничения}
\begin{itemize}
    \item \textbf{Проблема циклов}: Для каждого числа итераций создается новый путь
    \begin{codebox}
    for (int i = 0; i < n; i++)\\
    // Проблема: n - символьное значение\\
    // Экспоненциальный рост числа путей
    \end{codebox}
    
    \item \textbf{Вычислительная сложность}: Требует решения NP-полных задач
    \item \textbf{Ограничения SMT}: Строковые теории часто неполны \cite{SMTStrings}
    
    \item \textbf{Проблемы с памятью}: Символьные структуры данных сложны для анализа
\end{itemize}

\subsubsection{Оптимизации}
\begin{itemize}
    \item \textbf{Выборочная симвализация}: Только для критических переменных
    \item \textbf{Гибридные подходы}: Комбинация с абстрактной интерпретацией
    \item \textbf{Эвристики для циклов}: Ограничение глубины анализа
\end{itemize}

\newpage


\subsubsection{Абстрактная интерпретация}

\newpage


\subsubsection{Сравнение подходов}
\begin{center}
\begin{tabular}{|p{0.3\textwidth}|p{0.35\textwidth}|p{0.35\textwidth}|}
\hline
\textbf{Критерий} & \textbf{Символьное исполнение} & \textbf{Абстрактная интерпретация} \\
\hline
Точность анализа & Высокая (работает с конкретными значениями) & Средняя (аппроксимация) \\
\hline
Производительность & Низкая (комбинаторный взрыв) & Высокая (аппроксимация состояний) \\
\hline
Обработка циклов & Проблематична (экспоненциальный рост путей) & Эффективна (через неподвижные точки) \\
\hline
Применимость к строкам & Точный анализ конкретных строк & Работа с абстрактными шаблонами \\
\hline
Ложные срабатывания & Редко & Часто (из-за оверапроксимации) \\
\hline
\end{tabular}
\end{center}

\newpage



\subsection{Строковые абстрактные домены}


\specialsection{Выводы}

\newpage

\specialsection{Заключение}


\newpage

\specialsection{Благодарность}


\newpage
% Аргумент {1} ниже включает переопределенный стиль с выравниванием слева
\begin{thebibliography}{1}

\bibitem{CousotCousot77} Cousot P., Cousot R. \emph{Abstract Interpretation: A Unified Lattice Model for Static Analysis of Programs by Construction or Approximation of Fixpoints} // POPL. 1977.  
\url{https://doi.org/10.1145/512950.512973}

\bibitem{King76} King J.C. \emph{Symbolic Execution and Program Testing} // Communications of the ACM. 1976.  
\url{https://doi.org/10.1145/360248.360252}

\bibitem{ChristensenMøller03} Christensen A.S., Møller A. \emph{Precise Analysis of String Expressions} // SAS. 2003.  
\url{https://doi.org/10.1007/3-540-44898-5_10}

\bibitem{TARSIS} Arceri V. et al. \emph{Abstract Domains for String Analysis} // ACM Computing Surveys. 2022.  
\url{https://doi.org/10.1145/3494523}

\bibitem{SMTStrings} Zheng Y. et al. "SMT-Based String Analysis for Vulnerability Detection", 2021.

\bibitem{Z3} de Moura L., Bjørner N. "Z3: An Efficient SMT Solver" // TACAS. 2008.
\bibitem{CVC5} Barbosa H. et al. "cvc5: A Versatile and Industrial-Strength SMT Solver" // TACAS. 2022.

\end{thebibliography}


\end{document}